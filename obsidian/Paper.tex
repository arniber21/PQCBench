\documentclass[a4paper]{article}

\usepackage[T1]{fontenc}
\usepackage[utf8]{inputenc}
\usepackage{lmodern}

\usepackage[english]{babel}
\usepackage{csquotes}

\usepackage[notes,backend=biber]{biblatex-chicago}

\usepackage{amsfonts}
\usepackage{amsmath}
\usepackage{biblatex}
\bibliography{library}
\usepackage{tabularx}

\begin{document}
  \title{The Quantum Apocalypse: A Performance Evaluation of the
  NIST-selected Post-Quantum Cryptographic Algorithms with Big
  Payloads}
  \author{AP Research 2024}
  \maketitle

\section{Introduction}

\subsection{Cryptography}

Cryptography is the discipline concerning the protection of information.
Primarily, cryptography is used to secure communication between two
parties or to secure stored data. Cryptographers work to either
\emph{encrypt} information (obfuscate the message delivered from all
non-intended parties), \emph{decrypt} information (de-obfuscate the
information for the intended party), or communicate the obfuscated
message from one party to another in a secure manner. In general, we
define a cryptographic algorithm as an algorithm that achieves one or
more of the specified functions. We define a cryptographic scheme as a
set of cryptographic algorithms or a general approach for satisfying all
of these requirements \autocite{schneierAppliedCryptographyProtocols1996}.

When a cryptographic algorithm encrypts information, we call the message
we wish to encrypt the \emph{plaintext}. The algorithm successfully
encrypts this plaintext into a \emph{ciphertext}, which is obfuscated
from all non-intended parties. The intended party, then, decrypts the
ciphertext back to its plaintext form using a decryption algorithm
\autocite{lamCryptographyComputationalNumber2001}.

The vast majority of cryptographic schemes utilize a special set of
mathematical problems we define as \emph{intractable}. These intractable
problems are mathematical problems that are infeasible to solve without
some information, which we may call a key. The key is generated as some
part of the cryptographic scheme, then used by both parties to encrypt
or decrypt the desired message
\autocite{lamCryptographyComputationalNumber2001}.

\subsubsection[kem]{KEM}
For the
purposes of this paper, we shall consider a cryptographic scheme
(cryptosystem) as any KEM, or Key Exchange Mechanism. A KEM is a
corresponding set of three algorithms. There are many more complexities
to a KEM than specified in this paper; however, those details are not
necessary for the purposes of understanding this paper. The first
algorithm in a KEM is a key generation algorithm, which generates two
keys: a public key and a private key (also called a secret key).

The second KEM algorithm is called an encryption or encapsulation
algorithm. As the name suggests, the encryption algorithm obfuscates the
plaintext and returns a ciphertext. Crucially, this function takes the
aforementioned public key as an input. The third algorithm, then, is
called a decryption or decapsulation algorithm. This algorithm performs
the inverse of the encryption algorithm, taking both the generated
ciphertext and the private key.

\subsubsection[rsa]{RSA}
The most prominent cipher in
modern cryptography, RSA (named after its architects,
Rivest-Shamir-Adleman), utilizes the \emph{factoring into primes
problem}, which considers the prime factorization of very large numbers
\autocite{schneierAppliedCryptographyProtocols1996}. The scheme is named for
the three scientists who devised it at the Massachusetts Institute of
Technology in 1978. We define the factorization problem as follows:
given a positive integer \(x \in \mathbb{Z}^+\), find the unique set of
prime numbers \(p_1, p_2, p_3, \cdots p_n\) and integer exponents
\(k_1, k_2, k_3,\cdots,k_n\) such that
\(x = p_1^{k_1}p_2^{k_2}p_3^{k_3} \cdots p_n^{k_n}\). Interestingly, it
has been mathematically proven that there is \emph{no efficient way} to
factor large numbers; even the fastest methods are little more effective
than trial and error.

However, the factorization problem becomes trivial with a crucial bit of
information: just one of these factors and an associated exponent. Given
such a pair, we may simply divide the factor by x: \(p^k \cdot x^{-1}\)
in order to calculate the rest of the factorization trivially.
Logically, then, RSA creates a cryptosystem using the prime
factorization problem as defined below.\\

First, we pick two very large numbers \(p, q \in \mathbb{Z}\):
\[ p, q \in \mathbb{Z} \text{ (and very large) } \] \[ n := pq\] Then,
we generate a number \(e\) using this huge number \(n\). Note that
\(\lambda\) is any totient function, usually chosen as Carmichael's
totient but sometimes Euler's. A totient function is any function that
counts up the amount of prime numbers to an integer
\(n \in \mathbb{Z}^+\).
\[ 2 \leq e \leq \lambda(n) \text{ and } \text{gcd}(e, \lambda(n)) = 1\]
\(e\) can be chosen from random with the parameters above. Randomness is
an extremely important tool within cryptography. Note that it is
mathematically dubious to pick a number at random but the threat this
poses to literally any mathematical operation that is not purely
theoretical with no consequences is trivial. We set the public key to be
distributed freely as \((n, e)\). Finally, we can generate the private
key \(d\) as follows: \[ d := e^{-1} \mod{\lambda(n)} \] All of this
satisfies the first requirement of KEM: key generation.

 We are now ready to encrypt a
message. First, we have to take the given message and hash the message;
in other words, turn it into numbers.

Hashing the message has the secondary benefit of obfuscating the method
enough such that you can't derive the original message. While this may
seem counterintuitive, consider the case that we want to simply
\emph{verify} a message rather than send it. By hashing the message,
even if a third party were to get access to the hash somehow, they would
not be able to use it in any way. Password verification, for example,
utilizes hashes to this effect. However, if we do not wish to obfuscate
the message, we can use a simple hash such as Base64 that can be
inverted.

For mathematical purposes, define the hashing function
\(h: A \to \mathbb{Z}\) where \(A\) is the set containing the message.
\[ m := h(\text{message})\]
\[ c := m^e\mod{n} \text{  where } 0 \leq m < n\] We call \(c\) the
ciphertext, which we distribute in this case. Note that anyone can
generate the ciphertext, as long as they have the message they wish to
encrypt and the public key they wish to use.

After encrypting the message, we are now going
to decrypt the message. Remember that as the decrypting party, we have
access to three variables: 1. The ciphertext, \(c\) 2. The public key,
\((n, e)\). 3. The private key, \(d\). Also remember that the goal of
the algorithm is to derive \(m\), the original hashed message. RSA
allows us to compute this cleverly:\\
\[ c^d \equiv {m^e}^d \equiv m \mod{n}\] Notice that we did not generate
\(m\); rather, we found \(m \mod{n}\). However, because
\(0 \leq m < n\), \(m \mod{n} = m\). We have finished decrypting our
message, upon which we can either de-hash or verify.

You may have noticed a flaw with the above appraisal: we assume that
both parties already have the public and private keys. One may consider
that previously, we presumed most cryptosystems to be using a singular
key in order to crack the ciphertext. There is a problem with this
single-key approach, however: we must convey the key somehow to the
intended party without the prying eyes of unintended parties. In the
past, this was achieved through physical means beforehand; however, in
the modern era, most cryptosystems implement the Diffie-Hellman key
exchange algorithm (or something very similar) to achieve this.

The details of the Diffie-Hellman algorithm are not necessary to
understand this paper; simply recall that Diffie-Hellman results in the
distribution of \emph{two} keys: a public key and a private key. The
public key is freely distributed, and the private key is solely in the
hands of the two verified parties. The first algorithm of a KEM
cryptosystem generates these two keys.

\subsubsection[ecc]{Elliptic Curve Cryptography}
Recently, an alternative to RSA cryptosystems has arisen:
Elliptic-Curve Cryptography, or ECC for short. ECC relies on a different
mathematical construction: namely, the discrete logarithm problem
\autocite{schneierAppliedCryptographyProtocols1996}. Define the discrete
logarithm problem as follows: given a group \(G\) and generator
\(g \in G\) and element \(h \in G\), we can say that \(\log_g(h) = x\),
if \(x\) exists such that \(x\) is an integer and \(g^x = h\). To take
advantage of the discrete logarithm problem, elliptic curve cryptography
defines elliptic curves over finite fields. In doing so, they are able
to use much smaller key sizes than other algorithms while maintaining
similar or even improved levels of security, especially in
security-critical IoT infrastructure often afforded significant
performance hindrances \autocite{dhillonEllipticCurveCryptography2016,liuIoTNUMSEvaluatingNUMS2019}
.

Elliptic Curve Cryptography is largely considered preferential to RSA
and Diffie-Hellman based approaches to security. NIST has standardized
the set of elliptic curves in Federal Information Processing Standard(s)
(FIPS) 186, currently in its 5th iteration (note that the fourth
generation of elliptic curve standards set by FIPS 186-4 is set to be
retired in the early weeks of February 2024)
\autocite{DigitalSignatureStandard2023}.

With a basic understanding of the cryptographic primitives necessary for
this paper, we shall now explore the requisite quantum computing
terminology relevant to this paper.

\subsubsection[quantum]{Quantum Computing}
The vast majority of computers follow a classical computing model. In fact,
classical computing has become (rightfully so) synonymous with
computation as a whole. Classical computers are precise and exact: they
store data in a concrete and binary manner, in units of bits.
Computation occurs through logic gates and electrical circuits; these
calculations have little to no noise associated with them. As a result,
ironically, classical computers are slow: classical computers perform
jobs thoroughly, exactly, yet slowly.

The preciseness and reliability of classical computers, however, are
extremely useful. At scale, classical computers are still fast enough to
solve most problems demanded of them; additionally, algebraic and
computational theory has evolved to a sufficient maturity to which the
drawbacks associated with classical computers are not true drawbacks in
the majority of cases. However, due to their fundamentally thorough
nature, classical computers still remain limited.

Quantum computing, however, acts as a complement to classical computing
models. Fittingly taking advantage of the notoriously unpredictable
principles of quantum mechanics, such as superposition, quantum
computers are notoriously difficult to work with. Rather than classical
bits, which can either be 0 or 1 (denoting on or off on a circuit board,
and 0 or 1 mathematically), quantum computers store data in qubits,
which oscillate freely between 0 and 1 and may even simultaneously
occupy multiple values. Where classical computers emphasize accuracy and
preciseness, quantum computers are noisy and unpredictable. In fact, the
very act of attempting to gauge the value of a qubit can effect not only
its value, but the value of other qubits.

However, the seemingly paradoxical and imprecise nature of quantum
computing also belies its greatest strength. Because quantum computers
can store and process massive amounts of data simultaneously, they are a
great fit for applications such as scientific simulations, which often
do not require the accuracy of classical computers but desire the power
of quantum computers. For this, and many other tasks which often require
brute-force methods on classical computers, quantum computers show
promise as the compute ideology of the future. One such application,
named Shor's Algorithm, threatens nearly all modern security.

However, while quantum computers have matured at an extraordinary pace,
quantum computers are, compared to their classical counterparts,
infants. Even the largest quantum computers struggle to reach qubit
counts above 100, rendering current quantum computers useless for the
vast majority of applications. For this reason, most optimism regarding
quantum computing is limited to future possibilities. However, quantum
computers have grown exponentially in the past decade; with increasing
interest from government and private parties alike, all of which are
eager to unearth its power, quantum computing may become viable much
sooner than current estimates project.

\subsubsection[timecomplex]{Time Complexity}
As a small aside, we note that we may refer to terms such as linear or exponential
run time. This simply refers to the mathematical function class which
best approximates the time taken by an algorithm when given larger and
larger inputs. For example, an algorithm which runs in linear time has a
time signature approximated by a function of the class
\(T: x \mapsto ax + b\); an exponential algorithm has a time signature
of the class \(T: x \mapsto ba^x\). We can imagine the many different
types of time complexities which exist for many different algorithms.
Only elementary familiarity of time complexity is required for
comprehension of our experiment.

\section{Literature Review}\label{literature-review}

\subsection{Shor's Algorithm}\label{shors-algorithm}

Recall from the previous section that quantum computers pose an
existential threat to current cryptographic standards, compromising both
the RSA and ECC cryptosystems. This threat is due to an algorithm
devised by Mathematician Peter Shor in 1994; ironically, it remains the
only concrete feasible implementation of a quantum algorithm for a
useful problem.

Named Shor's Algorithm after its creator, the program solves the
aforementioned prime factorization and discrete logarithm problems
without the need for a key, thus bypassing all security guaranteed by
the two protocols. Shor's Algorithm solves both problems by solving a
related problem, the order finding problem, defined as follows for a
given positive integer \(a \in \mathbb{Z}^+\), and larger integer \(N\):
\[ \text{ord}_a = \text{ the smallest } r \text { where } a^r \equiv 1 \mod{N}\]
Utilizing some clever mathematics, one can convert both the discrete
logarithm and prime factorization problem to the order finding problem.
Shor's Algorithm, then, by finding a solution to the order finding
solution, by proxy, has found a solution to the other two problems. In
fact, Shor's Algorithm disqualifies a whole host of cryptographic
schemes; RSA and KEM are simply the most important.

With two of the state-of-the-art in cryptographic systems rendered
vulnerable to quantum-based attacks, it becomes vital that secure
post-quantum cryptography (PQC) is developed, immune to Shor's
Algorithm.

\subsubsection[quantumcryptography]{Quantum Cryptography}
Note that this paper concerns
\emph{post-quantum cryptography}, not quantum cryptography. Quantum
Cryptography is a separate field of cryptography concerning how to use
the properties and expanded capabilities of quantum computers to
\emph{produce} security- we are simply concerned with preventing quantum
computers from breaking existing security utilizing classical computers
\autocite{gisinQuantumCryptography2002}.

\subsection{Learning with Errors}\label{learning-with-errors}

In response to the threat presented by Shor's Algorithm, The National
Institute for Standards in Technology (NIST) held a public competition
to standardize a post-quantum cryptographic scheme resistant to Shor's
Algorithm. Of the four algorithms eventually selected to proceed by
NIST, three of the four algorithms utilized the \emph{Learning with
Errors} (LWE) problem in order to guarantee security. LWE effectively
combines two mathematical problems, the Closest Vector Problem (CVP) and
the Shortest Vector Problem (SVP) in order to guarantee security
\autocite{blanco-chaconRingLearningErrors2019}. SVP is defined as follows.
Note that the magnitude of a vector \(v\) is given with the following
formula \autocite{schneierAppliedCryptographyProtocols1996}:
\[ || v || = \sqrt{\sum_{i = 1}^n v^2_i}\] Now, given a lattice \(L\),
imagine finding a vector \(x \in L\) such that for every \(y \in L\),
the magnitude \(||x|| \leq ||y||\) - essentially, the smallest vector
problem. CVP is very similar: given a point \(y \in \mathbb{R}^n\), find
a vector \(x\) such that for all other vectors \(z \in L\),
\(|| y - x || \leq || y - z||\) - in essence, you are finding the
closest vector to a point.

LWE effectively combines both the CVP and SVP problems for a lattice of
many dimensions in order to guarantee security - while these problems
are feasible to solve in a few dimensions, adding dimensions to this
problem greatly increases its difficulty to a level considered
unbreakable by quantum computers with current knowledge.

However, LWE is also much less efficient than the state-of-the-art ECC
or RSA. Whereas ECC is commonly accepted as fast, viable, and secure
even in low performance devices
\autocite{dhillonEllipticCurveCryptography2016}, LWE is only approaching
\autocite{khalidLatticebasedCryptographyIoT2019,guillenPostquantumSecurityIoT2017,seyhanLatticebasedCryptosystemsSecurity2022}

 that designation.
However, early results are very promising in regards to CPU performance
\autocite{akleylekNewLatticebasedAuthentication2022}

\subsubsection[nist]{NIST} We note that in August 2023, NIST authored three reports
which outline the post-quantum cryptographic standards of all United
States federal communications: FIPS 203, FIPS 204. and FIPS 205
{[}@ThreeDraftFIPS2023{]}. These reports outline the use of Kyber in
particular as an encryption standard and outline the procedure for
creating a KEM using Kyber. In order to gauge the effectiveness of
post-quantum cryptography, we opt to use the standards outlined by NIST
in order to ensure the applicability of our experiment. The Kyber
specification itself outlined by PQ-Crystals may be adapted using a
variety of different implementations with different efficiencies; we opt
to use the most official specification via NIST.

\section{Methods}\label{methods}

We shall begin by stating the choices in the design of our experiment,
then following up this overview with a justification of these choices.

In order to gauge the feasibility of the NIST-selected post-quantum
algorithms, we opt to test two algorithms: Kyber and ECC-KEM (the
Elliptic Curve Integrated Encryption Scheme, or ECIES). Our approach is
purely quantitative in nature, collecting the following performance
metrics: CPU usage, as a percentage; memory usage, in megabytes; and
time, in seconds. We use a Haswell-based 12-core Intel Xeon processor;
however, we opt to test each algorithm on a single thread. Additionally,
in order to guarantee a level of separation from the base operating
system, we run each benchmark inside a Docker virtual machine.

In line with most cryptographic benchmarks, we choose a
quantitative study primarily to optimize ease of use. Quantitative
figures are easily available when measuring CPU and memory usage;
therefore, in order to take advantage of our mathematical toolkit to
analyze figures, we opt to use quantitative data. Crucially, we plan to
collect data at a scale in which manual data analysis is not feasible -
quantitative data allows us to computationally analyze these results
simultaneously to find the patterns we seek efficiently.

We primarily attempt to adhere to PQ-Crystals' own methodology
for benchmarking Kyber in the design of this experiment
\autocite{westerbaanX25519Kyber768Draft00HybridPostquantum2023}. However, we
must reconcile this goal with the reality that Kyber and ECIES are very
different algorithms; therefore, because we must ensure that there are
minimal differences in the benchmarking setup between the two
algorithms, we opt to build our own benchmarking script for both ECIES
and Kyber \autocite{liuIoTNUMSEvaluatingNUMS2019}. Let us describe the
individual benchmarking script for each algorithm a \emph{test saddle},
and relegate the benchmark label to the script producing the performance
data.

Our experiment, then, in line with other experiments, designs a test saddle and runs this test saddle with
a benchmark for each algorithm. We then compare the results for each of
the discrete combinations we wish to test. One may notice a large
discrepancy in our experiment compared to other experiments: whereas the
majority of other experiments measure CPU performance primarily through
cycles, we opt to measure CPU performance as a percentage of total CPU
usage. We recognize that there are several advantages to using cycles
over a percentage of total usage. Cycles are far more standardized and
accurate when compared percentages. However, recall that the focus of
this experiment is not to compute a direct performance comparison
between the two algorithms; rather, we wish to gauge how \emph{viable}
post-quantum algorithms are. Also recall that it is well established
that compute limitations on post-quantum algorithms are not limited even
by small CPUs; rather, we are far more concerned with memory usage. As a
consequence of testing two very different algorithms on Linux systems,
it therefore makes sense to prioritize the convenience of gathering
memory usage. In essence, we are sacrificing good CPU usage data in
order to gain more reliable and useful information in the form of memory
data.

\subsection{Procedures}\label{procedures}

It is true that NIST has selected other post-quantum algorithms: namely,
Dilithium; however, Dilithium and Kyber share the same mathematical
foundation (the Learning With Errors problem). If the purpose of this
study were to be a direct performance comparison between different
cryptographic algorithms, we would be obligated to test both Dilithium
and Kyber. However, because we solely wish to test the feasibility of
these algorithms rather than their direct performance, we conclude that
it is adequate to simply test Kyber. Additionally, PQ-Crystals, the
organization who developed both Kyber and Dilithium, provides direct CPU
performance comparisons under small payloads; therefore, a performance
comparison of the two is not necessary. Finally, note that we choose the
official implementation of Kyber
\autocite{hekkalaImplementingPostquantumCryptography2023}.

Also note that this study does not attempt to use multi-threading to
optimize either cipher. We choose not to take advantage of
multithreading primarily to objectify the study as much as possible.
This study does not make any assertions on the effect of effects of
multithreading on the effectiveness of either ECIES or KEM, nor does it
wish to. Because multithreaded approaches to neither are standardized,
we wish to relegate the potential effects of multithreading to future
studies and instead solely focus on the feasibility of the official
specifications of Kyber and ECIES. Attempting to multithread each would
introduce significant differences between the two, compromising any
attempt at a comparison
\autocite{ebrahimiPostQuantumCryptoprocessorsOptimized2019}. We elaborate on
this choice and its consequences in the limitations section of this
paper.

Earlier, we note that we use Docker as a virtualization platform. We
will briefly justify the use of a virtualization platform. A platform
such as Docker allows us to minimize performance overhead while
retaining a key advantage of virtualization: independence from the host
system. This allows each test to remain as independent and equal as
possible, with little to no difference in the starting
environments.\autocite{singhAdvancedLightweightEncryption2017} This can
remove sources of random bias, such as the operating system mysteriously
choosing to allocate more resources to one process over the other. The
primary disadvantage of virtualization is the significant performance
overhead induced by a virtual platform; however, Docker is minimal as to
minimize this effect to where the loss in performance is negligible.
Therefore, we opt to use Docker to increase the consistency of our
study.

Our benchmarking process generates CSV files
which report: a record number, the instantaneous time, the instantaneous
CPU usage, the instantaneous memory usage. We aggregate this data to
generate the following performance indicators for each combination of
payload and algorithm: time elapsed, minimum CPU usage, maximum CPU
usage, average CPU usage, minimum memory usage, maximum memory usage,
average memory usage. For special cases, Excel was used to generate
visuals to manually inspect trends. Truncated revisions of the raw data
collected will be provided in the Appendix; however, the size of this
data dictates that including all of the data simply is not feasible.

\section{Results}\label{results}

In the process of executing our experiment, we unexpectedly come to the
conclusion that presently, the application of post-quantum cryptographic
algorithms are, in fact, infeasible with large payloads. Furthermore, we
conclude that the official specifications of Kyber are fundamentally
incompatible with any task that wishes to operate on a larger than
normal file, and certainly infeasible even with large computational
resources.

Our de-facto control group, ECIES, produced as-expected performance
results. As show in Figure 1, with a maximum payload size of a sixteen
gigabyte random-byte scratch-file, memory follows a modest, linear
increase from around 192MB to 243MB. Note that we achieve such low
memory footprint using a segmented approach (splitting our large file
into many small files), which is considered standard in the encryption
of large files. We do so at the sacrifice of speed. We may nominally
consider that there are methods to expend more resources to optimize
this process; however, because we wish solely to obtain a benchmark for
which to compare Kyber, we choose not to make these optimizations. Make
note of this for future reference.

As a result, the elapsed time benchmarks for ECIES increased linearly:
with a half-gigabyte sized payload, the benchmark took 14 seconds; with
a one-gigabyte sized payload, the benchmark took 30 seconds; with a
two-gigabyte sized payload, the benchmark took 58 seconds; with a
sixteen-gigabyte sized payload, the benchmark too 487 seconds. In this
manner, segmentation is quite palatable because we are able to maintain
a linear time complexity. Note that a linear time complexity is not
necessarily always indicative of good performance; however, it is
generally a desirable label to attain because linear time complexities
scale excellently.
  \begin{table}[htbp]
      \centering
      \caption{Benchmark, Payload Size, and Time Elapsed}
      \label{tab:benchmark-payload-time}
      \begin{tabular}{@{}llll@{}}
          \toprule
          Benchmark & Payload Size (GB) & Time Elapsed (s) \\ \midrule
          ECIES     & 0.5               & 14               \\
          ECIES     & 1                 & 30               \\
          ECIES     & 2                 & 58               \\
          ECIES     & 4                 & 116              \\
          ECIES     & 8                 & 232              \\
          ECIES     & 16                & 487              \\
          Kyber     & 0.5               & 42339            \\ \bottomrule
      \end{tabular}
  \end{table}

  \begin{table}[htbp]
      \centering
      \caption{Benchmark, Payload Size, and CPU Stats}
      \label{tab:benchmark-payload-cpu}
      \begin{tabular}{@{}lllll@{}}
          \toprule
          Benchmark & Payload Size (GB) & CPU Min (\%) & CPU Max (\%) & CPU Avg (\%) \\ \midrule
          ECIES     & 0.5               & 0.13         & 3.9          & 2.4          \\
          ECIES     & 1                 & 0.12         & 3.9          & 2            \\
          ECIES     & 2                 & 0.14         & 4.9          & 1.5          \\
          ECIES     & 4                 & 0.13         & 5.2          & 4            \\
          ECIES     & 8                 & 0.13         & 5.4          & 3            \\
          ECIES     & 16                & 0.11         & 4            & 3.5          \\
          Kyber     & 0.5               & 0.14         & 4.3          & 3.8          \\ \bottomrule
      \end{tabular}
  \end{table}

  \begin{table}[htbp]
      \centering
      \caption{Benchmark, Payload Size, and Memory Stats}
      \label{tab:benchmark-payload-memory}
      \begin{tabular}{@{}lllll@{}}
          \toprule
          Benchmark & Payload Size (GB) & RAM Min (MB) & RAM Max (MB) & RAM Avg (MB) \\ \midrule
          ECIES     & 0.5               & 248           & 293           & 291           \\
          ECIES     & 1                 & 220           & 293           & 270           \\
          ECIES     & 2                 & 208           & 293           & 250           \\
          ECIES     & 4                 & 302           & 310           & 305           \\
          ECIES     & 8                 & 309           & 403           & 340           \\
          ECIES     & 16                & 192           & 503           & 451           \\
          Kyber     & 0.5               & 263           & 504           & 283           \\ \bottomrule
      \end{tabular}
  \end{table}

In line with memory usage, CPU usage using ECIES for the largest payload
remained in-line and relatively low, osc illating between less than one
percent and three percent usage as shown in Figure 1. Once again, we may
attribute this relatively low performance overhead to segmentation.
Perhaps most surprisingly, the performance impact of differently sized
payloads had little variation. This is, once again, most likely a
product of segmentation.

Our results begin to break down when inspecting Kyber's performance
data. For even a half-gigabyte payload, Kyber takes significantly longer
- 42339 seconds, or 11.76 hours - over 3024 times the amount of time
expended by ECIES. Kyber's significant time requirement dictated that
even for a single gigabyte, we could not feasibly run a full benchmark.
Therefore, we must conclude that the Kyber specification outlined for
use in FIPS 203-205 are not feasible for use with even medium-sized
payloads (the 16GB payload we planned to test), let alone the ones that
will be required to formulate a post-quantum transition (TB's of data).

\subsection{Analysis}
We reiterate that the current state of post-quantum
cryptography, at least that outlined by NIST, is infeasible with large
payloads. However, we note that this does not appear to be a purely
computational limit: while time usage was significantly higher for
Kyber, actual performance usage characteristics were not abnormal; in
fact, they were much lower than expected. Therefore, we speculate that
Kyber's infeasibility is a result of segmentation, as noted earlier.
Segmentation allowed the ECIES benchmarks to maintain low levels of
resource usage, in addition to slower than expected benchmarks; however,
we note that segmentation kept ECIES running in linear time. Even so, we
speculate that in real-world scenarios, the size of each segment would
expand in order to better utilize the available resources.

Upon further inspection of the FIPS protocols of which our official
Kyber implementation is based on, we find that Kyber internally uses
20-bit blocks in their encryption process, ensuring segmentation. 20
bytes is quite small; the mathematical implementation of Kyber requires
that bytes are this size in order to permit the usage of certain
operations that guarantee Kyber's security. We speculate that Kyber's
inefficiency originates from this 20-bit requirement: while certainly
usable for smaller payloads, as found in earlier experiments on smaller
computers, such segmentation significantly hinders performance on large
systems. Note that we do not craft our own implementation of Kyber to
specification; we use a pre-implemented official version of Kyber to the
FIPS specifications.

Furthermore, we may calculate that the 20 bit requirement required that
the half-gigabyte test which ran for 11.76 hours processed a single
20-bit block 25 million times, or processing close to 580 blocks per
second. We can conclude, therefore, that even for a much more capable
computer, allowing for a larger block size will be required to attain
feasibility on larger payloads. Until then, we may rest assured that
Kyber works wonderfully for smaller payloads, consuming a small
performance footprint. Additionally, the importance of FIPS 203-205
remains of the utmost importance to the post quantum transition, because
as prior research notes, it is quite feasible in low-performance
environments in scenarios which segmentation is not problematic.

\section{Conclusions}\label{conclusions}

\subsection{Segmentation}\label{segmentation}

We come to the unfortunate conclusion that we, unfortunately, cannot
conclude that the current NIST standards are feasible for the encryption
of larger payloads. In fact, the feasibility for even medium-sized
payloads are not currently feasible. While post-quantum cryptographic
algorithms may be adequate for most use cases with smaller payloads ,
significant progress and perhaps a complete retool is required to
facilitate usability with big data. We speculate in the analysis section
that segmentation makes up the majority of this problem; the current
block sizes outlined by NIST for Kyber are too small. Perhaps a larger
block, at the cost of using more resources, could be used to facilitate
the usage of Kyber with larger payloads. However, it is important to
note that the underlying implementation of many cryptographic algorithms
rely on blocks of small sizes in order to perform a desired mathematical
operation. The security of these algorithms are contingent on these
mathematical operations; thus, increasing the block size may be a tall
order and may require the development of an entirely new algorithm, or
even entirely disqualify certain approaches. Therefore, we fail to make
a conclusion on the merits of segmentation.

\subsection{Multithreading}\label{multithreading}

Allow us to consider the case of a medium-sized data-transition: a 10TB
database. Let us label the efficiency of our benchmark \(\gamma_0\).
Note that we tested on a 0.5GB sized payload; therefore, assuming a
linearly segmented encryption process, we can speculate that our task
will take \(\frac{10TB}{0.5GB} = 20,000\) times longer than our
benchmark assuming a constant efficiency of \(\gamma_0\), we arrive at
235,200 hours: or, 9,800 days which is nearly 27 years. If we were to
assume that our task could run on multiple threads; for the sake of
argument, 64, we still require \textasciitilde154 days of continuous
data processing. Additionally, this assumes a strict, best-case scenario
where the performance benefits afforded by multithreaded
parallel-processing occurs as efficiently as possible.

Earlier, we stated that we do not attempt to multithread in this study.
We do not multithread precisely because the performance benefits
afforded by parallelism and multithreaded processing are not completely
efficient. In other words, allowing our process 2 cores compared to one
core would not necessarily double our performance; in fact, in some
cases, multithreading can actually decrease performance. While we may
hypothesize that our experiment would have been much faster had we
decided to multithread, the applicability of our results would be
compromised as our specific implementation of multithreading would not
have been official. In fact, we recommend further research exploring the
efficacy of multithreading in post-quantum encryption schemes; a general
idea of the performance benefits yielded by multithreading could pave
the way for future development of a more efficiently multithreaded
post-quantum encryption algorithm in order to guarantee effectiveness.

\subsection{Limitations} There are many limitations associated with our
interpretation of this question. For example, we use a quite old
hardware setup that may have hardware constraints. Additionally, we do
not use multithreading; this is elaborated on in the previous section.
In respect to algorithms, we opt only to test Kyber; however, different
post-quantum algorithms may have different results (though that is
unlikely given that they all utilize the same mathematical framework).
Additionally, we tested only one payload with Kyber due to feasibility
issues; a more powerful setup may have been able to compute a result for
a larger Kyber benchmark that could be used to produce additional
conclusions regarding Kyber's performance at scale. We will examine each
limitation in full and provide an explanation for each.

We are unfortunately limited by the computational resources at our
disposal. We choose a notably older Intel Haswell architecture with an
older Xeon processor. Therefore, it may not be completely accurate to
extrapolate conclusions on a Haswell CPU to the modern CPUs that the
majority runs on today. However, while this is true, it is unlikely that
purely architectural differences play a major role in the validity of
our conclusions. Rather, our resource limitations prevent us from
speculating how performance may scale up. Perhaps a newer processor
might have demonstrated a greater ability to scale with larger payloads
than our Haswell processors, via features such as a larger cache of
improved multithreading or otherwise.

One may notice that we only experimentally conclude infeasibility for a
single post-quantum cryptographic algorithm: Kyber. We choose to do so
because it is the algorithm specifically denoted by NIST in FIPS 203 -
205 \autocite{ThreeDraftFIPS2023}. However, doing so does create further
questions on perhaps the increased scalability or feasibility of other
algorithms, perhaps an algorithm which relies less on segmentation.
Therefore, it may be advisable for future research to study the
performance of other cryptographic algorithms with larger payloads, in
addition to exploring the effects of multithreading. Combined with a
future study on segmentation in particular, one my find differing
conclusions than ours regarding the feasibility of a post-quantum
algorithm.

\section{Bibliography}

\printbibliography

\end{document}
